\section{Выводы}

В этой работе я на практике реализовал стабильную сортировку подсчётом не в
виде на весь диапазон, а в более аккуратном варианте - с учётом
реально встречающихся ключей. В реальности, такое небольшое изменение
реализации (учёт минимального и максимального ключа) может снизить расход
памяти на определённом наборе входных данных, однако далеко не всегда.

В итоге сортировка подсчётом действительно хорошо подходит для задач
с целочисленным ключом из ограниченного диапазона. Однако при значительном
расширении диапазона ключей алгоритм деградирует сразу по двум параметрам:
и расход памяти, и время работы растут как $O(k)$. Если диапазон $k$
существенно превышает количество элементов n, сортировка подсчётом
проигрывает обычным сравнительным алгоритмам с $O(n log n)$.
\pagebreak